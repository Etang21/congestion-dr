% JEL-Article.tex for AEA last revised 22 June 2011
\documentclass[JEL]{AEA}

% The mathtime package uses a Times font instead of Computer Modern.
% Uncomment the line below if you wish to use the mathtime package:
%\usepackage[cmbold]{mathtime}
% Note that miktex, by default, configures the mathtime package to use commercial fonts
% which you may not have. If you would like to use mathtime but you are seeing error
% messages about missing fonts (mtex.pfb, mtsy.pfb, or rmtmi.pfb) then please see
% the technical support document at http://www.aeaweb.org/templates/technical_support.pdf
% for instructions on fixing this problem.

% Note: you may use either harvard or natbib (but not both) to provide a wider
% variety of citation commands than latex supports natively. See below.

% Uncomment the next line to use the natbib package with bibtex 
\usepackage{natbib}

% Uncomment the next line to use the harvard package with bibtex
%\usepackage[abbr]{harvard}

% This command determines the leading (vertical space between lines) in draft mode
% with 1.5 corresponding to "double" spacing.
\draftSpacing{1.5}

\begin{document}

\title{Inequality in Market Design}
\shortTitle{(Exploration of Open Questions)}
\author{Eric Tang\thanks{%
Acknowledgements}}
\date{\today}
\pubMonth{Month}
\pubYear{Year}
\pubVolume{Vol}
\pubIssue{Issue}
\JEL{}
\Keywords{}

\begin{abstract}
The work and open directions below focus on how we can harness market design to achieve more equitable outcomes for market participants. Some applications include rent control policies in housing markets, price floors in Iran's kidney exchange (\cite{dworczak-2020}), and allocation of food to food banks (\cite{prendergast-2017}).
\end{abstract}

\maketitle

Reading through more of the literature on matching markets, combinatorial assignment, and market design more broadly, I've become more interested in the potential to design markets with equity as a primary objective. These works below all link to that goal, despite being applied in somewhat different settings. This appears to be a growing field of work, with a pending in anthology in Kominers and Teytelboym's forthcoming ``More Equal by Design: Economic Design Responses to Inequality".

\section{Redistribution through Markets}

Price regulations, such as rent control and minimum wage laws, typically face tradeoffs between reducing inequality and maintaining efficient outcomes. \cite{dworczak-2020} characterize the Pareto frontier of this tradeoff by framing price regulation as a market design problem, with buyers and sellers seeking to buy an indivisible good. They frame the Pareto frontier as determining the welfare-maximizing solution when agents have different valuations for money and for the good. (Roughly speaking, the valuation for money falls with an agent's wealth). They find that when there is significant inequality between buyers and sellers, optimal price controls take the form of a ``tax" which charges buyers a lower price than that of sellers, then transfers the surplus to the poorer side of the market. When there is significant inequality between agents on one side of the market, price controls take a more complex form but may still be optimal.

One interesting novelty of the \cite{dworczak-2020} model is their use of agents' differing marginal values for money, instead of the more standard approach of giving agents different Pareto weights to construct the Pareto frontier. The authors show that these two formulations are actually equivalent. Intuitively, an agent having a higher marginal utility for money is akin to a market designer placing greater weight on the money they are given in the final allocation. They show that agents' actions are fully characterized by their rate of substitution between the good and money.

The bulk of the paper describes the optimal mechanisms to address cross-side inequality (between sellers and buyers) and same-side inequality (between agents on the same side of the market). The paper discusses applications in markets for kidney exchange (rather, the one market which runs in Iran), which is generally an example of same-side inequality among sellers. It also discusses applications in the housing rental market, which is generally an example of cross-side inequality.

% Possible TODO: Could better-explain the mechanisms used, and the case studies examined.

The paper raises several interesting extensions. Most immediately, the model only studies an indivisible good without wealth effects. Both of these modeling assumptions could be extended. The model also neglects possible aftermarkets of trade between agents, which could be explicitly modeled. Finally, the rationing mechanism used to address same-side inequality is randomized; work on designing non-random mechanisms could be fruitful. It would also be interesting to find real-world examples where this randomization impacts participant behavior in measurable ways.

\section{Fair Equilibrium in Combinatorial Assignments}

\cite{budish-2011} opened exploration of the combinatorial assignment problem, in which a market designer must allocate indivisible bundles to agents, and in which monetary transfers are prohibited. Critically, the market designer cares about both the efficiency and fairness of the allocation. This problem becomes interesting to study because the only efficient and strategyproof mechanism is a serial dictatorship which is (no surprise) not at all fair. Budish first defines two fairness criteria for the problem, then proposes a mechanism which achieves fairness with relaxations of the efficiency and fairness requirements.

The two fairness conditions, ``maximin share guarantee" and ``envy bounded by a single good," generalize existing fairness notions to the indivisible goods case. The proposed mechanism, the ``Approximate Competitive Equilibrium from Equal Incomes" mechanism, guarantees fairness when incomes are arbitrarily close (but not equal), and when the market designer can tolerate an error in the market clearance. Budish then shows that this satisfies a relaxation of the strategyproof condition. Finally, Budish illustrates that the mechanism empirically improves on existing methods to allocate courses to students.

\cite{babaioff-2019} generalize the results of \cite{budish-2011} to agents with budgets that can be far from equal. Here, the intuition is that in some cases, the designer may wish to favor some agents' preferences over others. For example, in \cite{budish-2011}, all students have the same priority to the market designer. In \cite{babaioff-2019}, the authors suggest the motivating example of a charity allocating food to food banks of different sizes, which should likely have different budgets.

\cite{prendergast-2017} describes the implementation of a similar combinatorial assignment mechanism for Second Harvest, a charity which distributes food to markets. The charity faced a combinatorial assignment problem in distributing its goods to affiliated food banks, and implemented a system similar to \cite{budish-2011}, by using artificial currency and having food banks bid on items.

\section{Open Questions}

\begin{enumerate}
    \item In cases of significant same-side inequality, the optimal mechanism proposed by \cite{dworczak-2020} is a randomized mechanism, in which some agents can only trade with a given probability. As the authors point out, randomized rationing in practice can be particularly harmful to poorer individuals, who may have less tolerance for uncertainty. There is also a long tradition in computer science of trying to "de-randomize" randomized algorithms. \textbf{Can we model the harm to poorer agents in this model that results from randomization? Can we lessen this impact with a different randomized mechanism, or derandomize the mechanism entirely? How might these answers change if the good is divisible?}
    
    \item \cite{budish-2011} addresses the combinatorial assignment problem, in which agents are allocated bundles of indivisible objects, monetary transfers are prohibited, and the market designer cares about efficiency and equity. Budish proposes the Approximate CEEI algorithm to resolve the problem, and empirically tests its results on student course assignments. Yet the only applications of this mechanism, to my knowledge, have been in Budish's course allocation and \cite{prendergast-2017} applying the mechanism to food distribution for food banks. Budish only alludes to one other applications, that of shift scheduling for workers in a firm. Yet it still feels that there are other killer applications of fairness in combinatorial assignment to be tapped. \textbf{What other applications of combinatorial assignment problems might be out there?}
    
    % \item (Still a little half-baked). \cite{babaioff-2019} explores an example in which agents have different budgets, motivated by food banks of different sizes. This is somewhat motivated by the market designer's preferences, in which each food bank should have different demand based on its size. One could also set these budets based on the market designer's own interests: for example, a distant food bank may have a lower budget due to the higher costs of transporting supplies to it. Motivated by this (admittedly somehwat contrived) example, suppose that the market designer has differential transport costs by the type of good in a bundle. I believe (could be wrong) this is akin to the market designer setting different prices for different agents, effectively price discriminating with respect to the vitual currency. \textbf{If the market designer may set different prices for different agents, in addition to allocating agents' different budgets, can we obtain novel outcomes? What applications might this have in real markets?}
\end{enumerate}

% Remove or comment out the next two lines if you are not using bibtex.
\bibliographystyle{aea}
\bibliography{matching}

% The appendix command is issued once, prior to all appendices, if any.

\end{document}

