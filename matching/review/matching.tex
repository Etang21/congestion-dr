% JEL-Article.tex for AEA last revised 22 June 2011
\documentclass[JEL]{AEA}

% The mathtime package uses a Times font instead of Computer Modern.
% Uncomment the line below if you wish to use the mathtime package:
%\usepackage[cmbold]{mathtime}
% Note that miktex, by default, configures the mathtime package to use commercial fonts
% which you may not have. If you would like to use mathtime but you are seeing error
% messages about missing fonts (mtex.pfb, mtsy.pfb, or rmtmi.pfb) then please see
% the technical support document at http://www.aeaweb.org/templates/technical_support.pdf
% for instructions on fixing this problem.

% Note: you may use either harvard or natbib (but not both) to provide a wider
% variety of citation commands than latex supports natively. See below.

% Uncomment the next line to use the natbib package with bibtex 
\usepackage{natbib}
\usepackage{hyperref} 

% Uncomment the next line to use the harvard package with bibtex
%\usepackage[abbr]{harvard}

% This command determines the leading (vertical space between lines) in draft mode
% with 1.5 corresponding to "double" spacing.
\draftSpacing{1.5}

\begin{document}

\title{Inequality in Market Design}
\shortTitle{(Exploration of Open Questions)}
\author{Eric Tang\thanks{%
Acknowledgements}}
\date{\today}
\pubMonth{Month}
\pubYear{Year}
\pubVolume{Vol}
\pubIssue{Issue}
\JEL{}
\Keywords{}

\begin{abstract}
The work and open directions below focus on how we can harness market design to achieve more equitable outcomes for market participants. Some applications include rent control policies in housing markets, price floors in Iran's kidney exchange (\cite{dworczak-2020}), and allocation of food to food banks (\cite{prendergast-2017}).
\end{abstract}

\maketitle

Reading through more of the literature on matching markets, combinatorial assignment, and market design more broadly, I've become more interested in the potential to design markets with equity as a primary objective. These works below all link to that goal, despite being applied in very different settings.

\section{Redistribution through Markets}

Price regulations, such as rent control and minimum wage laws, typically face tradeoffs between reducing inequality and maintaining efficient outcomes. In an age of dramatic inequality (see: anyone, perhaps Piketty most specifically), it is critical to design markets with equity in mind.

Early work exploring this balance includes \cite{weitzman-1977}, who asks whether competitive prices or rationing in a one-commodity market is more effective at ensuring agents with the highest need receive the good. Weitzman focuses primarily on goods which are deemed a societal need, potentially including housing, food, or medical care. As in actual marketplaces, agents have both differing incomes and differing needs for the good. Intuitively, if income distribution is relatively egalitarian and agents' needs vary widely, then competitive prices better identifies those agents with greater need. On the other hand, if income inequality is high compared to the dispersion in need for the good, rationing is more effective.

Formally, \cite{weitzman-1977} reach these results by parametrizing agents' demand functions according to $\lambda$ and $\epsilon$, where $\lambda$ represents the agent's marginal value for money (falls as income rises) and $\epsilon$ represents the agent's need for the good. Critically, these types are unobservable by the market designer, a problem which will arise again in later work as well.

Weitzman now considers two possible rules for allocating a fixed quantity of the good to each agent. Under competitive prices, the market designer sets a price to ensure market clearing. Under pure rationing, the market designer equally divides the good between all agents. Note that \cite{dworczak-2020} build on this model by, among other improvements, considering more nuanced rationing schemes that allocate to a subset of agents or impose multiple price levels. Weitzman then may compute a loss for each method in meeting actual need, and compare the two schemes' outcomes based on the prior distributions of $\lambda$ and $\epsilon$ in the populace. This formalizes the intuition of when rationing or competitive prices better meet needs in the market.

\cite{dworczak-2020} characterize the Pareto frontier of this tradeoff by framing price regulation as a market design problem, with buyers and sellers seeking to buy an indivisible good. They frame the Pareto frontier as determining the welfare-maximizing solution when agents have different valuations for money and for the good. (Roughly speaking, the valuation for money falls with an agent's wealth). They find that when there is significant inequality between buyers and sellers, optimal price controls take the form of a ``tax" which charges buyers a lower price than that of sellers, then transfers the surplus to the poorer side of the market. When there is significant inequality between agents on one side of the market, price controls take a more complex form but may still be optimal.

The key obstacle which \cite{dworczak-2020} overcomes, as was faced by earlier models such as Weitzman (1977), is that the market designer does not perfectly identify agents' needs, and hence setting a price at which the commodity is rationed helps to identify these needs.

One interesting novelty of the \cite{dworczak-2020} model is their use of agents' differing marginal values for money, instead of the more standard approach of giving agents different Pareto weights to construct the Pareto frontier. The authors show that these two formulations are actually equivalent. Intuitively, an agent having a higher marginal utility for money is akin to a market designer placing greater weight on the money they are given in the final allocation. They show that agents' actions are fully characterized by their rate of substitution between the good and money.

The bulk of the paper describes the optimal mechanisms to address cross-side inequality (between sellers and buyers) and same-side inequality (between agents on the same side of the market). The paper discusses applications in markets for kidney exchange (rather, the one market which runs in Iran), which is generally an example of same-side inequality among sellers. It also discusses applications in the housing rental market, which is generally an example of cross-side inequality.

% Possible TODO: Could better-explain the mechanisms used, and the case studies examined.

The paper raises several interesting extensions. Most immediately, the model only studies an indivisible good without wealth effects. Both of these modeling assumptions could be extended. The model also neglects possible aftermarkets of trade between agents, which could be explicitly modeled. Finally, the rationing mechanism used to address same-side inequality is randomized; work on designing non-random mechanisms could be fruitful. It would also be interesting to find real-world examples where this randomization impacts participant behavior in measurable ways.


% Interesting! In the Redistributive Allocation paper, the authors mention that they are planning on applying the work to vaccines in an upcoming paper.

\cite{akbarpour-2020} approach the problem slightly differently, as one in which a market designer is alllocating a public good of heterogeneous quality to buyers, and additionally deciding on the price. Also notable about the \cite{akbarpour-2020} model is that it weights the valuation of generating revenue with its own Pareto weight, reflecting the potential uses of that generated revenue. Finally, the work also introduces observable types (think of observable demographic characteristics) which the market designer can condition on.


\section{Fair Equilibrium in Combinatorial Assignments}

\cite{budish-2011} opened exploration of the combinatorial assignment problem, in which a market designer must allocate indivisible bundles to agents, and in which monetary transfers are prohibited. Critically, the market designer cares about both the efficiency and fairness of the allocation. This problem becomes interesting to study because the only efficient and strategyproof mechanism is a serial dictatorship which is (no surprise) not at all fair. Budish first defines two fairness criteria for the problem, then proposes a mechanism which achieves fairness with relaxations of the efficiency and fairness requirements.

The two fairness conditions, ``maximin share guarantee" and ``envy bounded by a single good," generalize existing fairness notions to the indivisible goods case. The proposed mechanism, the ``Approximate Competitive Equilibrium from Equal Incomes" mechanism, guarantees fairness when incomes are arbitrarily close (but not equal), and when the market designer can tolerate an error in the market clearance. Budish then shows that this satisfies a relaxation of the strategyproof condition. Finally, Budish illustrates that the mechanism empirically improves on existing methods to allocate courses to students.

\cite{babaioff-2019} generalize the results of \cite{budish-2011} to agents with budgets that can be far from equal. Here, the intuition is that in some cases, the designer may wish to favor some agents' preferences over others. For example, in \cite{budish-2011}, all students have the same priority to the market designer. In \cite{babaioff-2019}, the authors suggest the motivating example of a charity allocating food to food banks of different sizes, which should likely have different budgets.

\cite{prendergast-2017} describes the implementation of a similar combinatorial assignment mechanism for Second Harvest, a charity which distributes food to food banks. The charity faced a combinatorial assignment problem in distributing its goods to affiliated food banks, and implemented a system similar to \cite{budish-2011}, by using artificial currency and having food banks bid on items.

\section{Miscellaneous}

Here are topics that I find interesting, which are sorted by topic but not by market ``type'', as the problems above are.

\subsection{School Choice}

Perhaps the application of market design most directly motivated by alleviating inequality has been school choice. \cite{avery-2020} do very interesting work to extend models of school choice, motivated by the idea that un-tethering schools from their neighborhoods changes neighborhood housing prices. Intuitively, school choice tends to equalize the performance of schools within a city, resulting in a compressed spectrum of housing prices. They then uncover an ``ends-against-the-middle'' tradeoff, wherein families at the high and low ends of the socioeconomic spectrum have an incentive to leave the town. In particular, even when the lowest-performing schools improve, low-income individuals may not benefit. For an early overview of school choice mechanisms, \cite{pathak-2011} was helpful.

\subsection{Vaccine Distribution}

\cite{pathak-2020} apply market design principles to the distribution of necessities such as vaccines and ventilators during the COVID-19 pandemic. They argue that the standard system in which states allocate vaccines, in which all individuals are ordered in one queue of priority, can harm disadvantaged communities, and often leads to discarding several ethical dimensions for the sake of ordering all individuals in one queue.

They instead propose a system of ``reserves'', in which a number of items (in our case, vaccines) are allocated to each category of priority individuals. For example, 20\% of doses might be allocated to minority communities, 20\% to first-responders, and the remaining 60\% be open to all. \cite{pathak-2020}  identify several axioms that any such system should satisfy, then fully characterize these reserve systems. They show that these resserve systems are cutoff equilibria, and link these characterizations to the classic deferred-acceptance algorithm. 

Most interesting in the paper is the consequences of their work for underprivileged groups who have their own reserve allotted; the authors characterize which reserve equilibria are likely to be best for these groups. This paper is quite impressive in that \cite{pathak-2020} have advised Massachusetts and Tennesse in revising their vaccine allotment systems in just the past year. \cite{pathak-2020b} simulate current vaccine distribution policy to forecast the share of vaccines that will be delivered to Black and Indigenous communities. They compare this share of vaccines to a reserve system in which 40\% of vaccines are reserved for socioeconomically disadvantaged communities, under their reserve system proposed in \cite{pathak-2020}. Indeed, the proposed change delivers more vaccines to Black and Indigenous communities.  

\section{Other Papers to Read}

\begin{itemize}
    \item Scheuer, F. (2014): “Entrepreneurial taxation with endogenous entry,” American Economic Journal: Economic Policy, 6, 126–63.
            
    \item Maybe: Saez, E., and Stantcheva, S. (2016). Generalized social marginal welfare weights for optimal tax theory. American Economic Review, 106(1), 24-45.
    
    \item Kominers, S. and Teytelboym, A.: ``More Equal by Design: Economic Design Responses to Inequality". (Forthcoming book).
    
    \item Condorelli 2013.
    
    \item Work by Nathan Hendron on public finance.

\end{itemize}

\section{Miscellaneous Ideas to Follow}

(Some scattered ideas that I'm still working through).

\begin{itemize}
    \item Alternative rationing devices: how can we ration in \cite{dworczak-2020}, say, to ensure better ex-post fairness? What if this allocation is repeated several times?
    
    \item How does redistribution in a market in the sense of \cite{dworczak-2020} interact with other redistributive mechanisms like taxation? What if this market also had a progressive income tax imposed, in which the tax agent knew agents' incomes but the market designer did not?
    
    \item Other combinatorial assignment ideas: allocating vaccines or ventilators, allocating a set of seats on flights, reserving seats on public transit. Could you imagine allocating PPE to various hospitals as a combinatorial assignment problem?
    
    \item \cite{akbarpour-2020} value revenue generation by attaching it its own Pareto-weight, representing how the market designer can use that revenue. Can we model revenue redistribution more explicitly? Does that buy us anything?
    
    \item The importance of revenue generation in \cite{akbarpour-2020} (optimal mechanisms differ based on the Pareto weight of revenue generation) reflects the similar importance of that revenue generation witnessed in most congestion pricing models. Often, how the revenue is spent almost completely determines whether a congestion toll is progressive or regressive. Can we apply frameworks like that of \cite{akbarpour-2020} to congestion pricing? Unlike the intended use, though, this isn't quite distributing an "essential" good, as many of these motivating examples do.
\end{itemize}

\section{Open Questions}

I still need to deepen the \emph{specificity} of these questions, I think, and a better sense of how to sharpen good theory questions. I still need to develop a more crisp understanding of how \cite{dworczak-2020} differs from the prior literature on optimal taxation and rationing.

\begin{enumerate}    
    \item \cite{budish-2011} addresses the combinatorial assignment problem, in which agents are allocated bundles of indivisible objects, monetary transfers are prohibited, and the market designer cares about efficiency and equity. Budish proposes the Approximate CEEI algorithm to resolve the problem, and empirically tests its results on student course assignments. Yet the only applications of this mechanism, to my knowledge, have been in Budish's course allocation and \cite{prendergast-2017} applying the mechanism to food distribution for food banks. Budish only alludes to one other applications, that of shift scheduling for workers in a firm. Yet it still feels that there are other killer applications of fairness in combinatorial assignment to be tapped. \textbf{What other applications of combinatorial assignment problems might be out there?}
    
    \item Avery and Prathak (2020) model a city using school choice assignment of students, and complicate previous models by modeling (a) the effect of school quality on neighborhood housing prices and (b) an outside option where residents can move to a different city with neighborhood-based assignment. They find an “ends-against-the-middle” tradeoff in which both low-income and high-income types leave the city after it adopts school choice. Intuitively, implementing school choice compresses the range of school qualities, thus compressing the range of housing prices as well, and inducing those at the high and low ends of the spectrum to leave. \textbf{Do we actually observe high and low income types leaving cities after the adoption of school choice? To what extent?} This is motivated by some studies (one from the NBER labor conference which I need to find), showing that workers don’t follow idealized perfect geographic mobility models (e.g. residents in manufacturing towns hit by NAFTA tended to stay there, despite economic pressures). On the other hand, authors like Baum-Snow and Luntz (2010) find many wealthy families leaving Boston in response to desegregation there.
    
    \item In cases of significant same-side inequality, the optimal mechanism proposed by \cite{dworczak-2020} is a randomized mechanism, in which some agents can only trade with a given probability. As the authors point out, randomized rationing in practice can be particularly harmful to poorer individuals, who may have less tolerance for uncertainty. There is also a long tradition in computer science of trying to "de-randomize" randomized algorithms. \textbf{Can we model the harm to poorer agents in this model that results from randomization? Can we develop a mechanism that ``trades off'' efficiency for less randomization? How might these answers change if the good is divisible?} April 5: to put this in more formal terms, perhaps the question is that of ex-post versus ex-ante fairness. If this game is repeated, can we explicitly choose which agents are rationed to reduce ex-post unfairness? If agents are risk-averse, how do their preferences change?
    
    \item One timely, fairly direct analogy to the treatment of \cite{prendergast-2017} of food banks is the question that arose in 2020 of how to distribute Personal Protective Equipment (PPE) from the Strategic National Stockpile to hospitals which needed it. Here, as in the case of Feeding America, the equipment to be distributed were bundles of items (not funds), and equity concerns were a high priority. As in the prior arrangement for Feeding America, the distribution of the Strategic National Stockpile resulted in some inconsistent allocations (one example being Washington State \href{governor.wa.gov/news-media/we’re-together-–-washington-state-send-ventilators}{returning several hundred ventilators to other states}). On the other hand, the goods in the Stockpile are perhaps less heterogeneous than shipments of differing foods in \cite{prendergast-2017}. It's also not clear to me whether states were granted PPE shipments or had to purchase them at some minimal cost (merits investigation). The last concern is that I'm not sure how much novelty this would add to the similar implementations of \cite{prendergast-2017} and \cite{budish-2011}. Feasibility: decent, novelty: possibly low?, bearing on real world: reasonably high.
    
    \item For a fairly technical extension, in the accompanying seminar talk for \cite{akbarpour-2020}, Piotr Dworczak notes that their work only considers the case when there are a finite number of different Pareto weights, but their work could likely be extended to the continuous case (when Pareto weights are continuous). This could be an interesting technical extension, but I'm not sure what the real-world analogue of interest would be here.

    
    % \item (Still a little half-baked). \cite{babaioff-2019} explores an example in which agents have different budgets, motivated by food banks of different sizes. This is somewhat motivated by the market designer's preferences, in which each food bank should have different demand based on its size. One could also set these budets based on the market designer's own interests: for example, a distant food bank may have a lower budget due to the higher costs of transporting supplies to it. Motivated by this (admittedly somehwat contrived) example, suppose that the market designer has differential transport costs by the type of good in a bundle. I believe (could be wrong) this is akin to the market designer setting different prices for different agents, effectively price discriminating with respect to the vitual currency. \textbf{If the market designer may set different prices for different agents, in addition to allocating agents' different budgets, can we obtain novel outcomes? What applications might this have in real markets?}
\end{enumerate}

% Remove or comment out the next two lines if you are not using bibtex.
\bibliographystyle{aea}
\bibliography{matching}

% The appendix command is issued once, prior to all appendices, if any.

\end{document}

